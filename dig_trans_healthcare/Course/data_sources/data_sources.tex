\documentclass[10pt, xcolor=table]{beamer}

\setbeamertemplate{note page}[default]
%\setbeameroption{hide notes}
\setbeameroption{show notes}
\setbeamerfont{footnote}{size=\tiny}

\usetheme[progressbar=frametitle]{metropolis}
\usepackage{appendixnumberbeamer}

\usepackage{booktabs}
\usepackage[scale=2]{ccicons}

\usepackage{pgfplots}
\usepgfplotslibrary{dateplot}
\usepackage{multicol}
\setlength{\columnsep}{1.5cm}
\usepackage{multirow}

\usepackage{hyperref}
%\usepackage{animate}
\usepackage{lmodern}
\usepackage[T1]{fontenc}
\usepackage{mathtools}
\usepackage{graphicx}
\usepackage[font=scriptsize]{caption}
\usepackage{tikz}
\usepackage{stackengine}
\usepackage{array}
\usetikzlibrary{positioning}
\usepackage{tabularx}
\usepackage{tabulary}
%\hypersetup{
%    colorlinks=true,
%    linktoc=none,
%    linkcolor=blue,
%    urlcolor=blue
%}

\usepackage[math]{cellspace}
\cellspacetoplimit 2pt
\cellspacebottomlimit 2pt


%\definecolor{set1}{RGB}{228, 26, 28}
%\definecolor{set2}{RGB}{77, 175, 74}
%\definecolor{set3}{RGB}{255, 127, 0}
%\definecolor{set4}{RGB}{166, 86, 40}
%\definecolor{set5}{RGB}{153, 153, 153}

\usepackage{xspace}
\newcommand{\themename}{\textbf{\textsc{metropolis}}\xspace}

\newcommand\Fontvi{\fontsize{8}{9}\selectfont}
\newcommand\Fontvr{\fontsize{6}{7}\selectfont}

\setbeamerfont{parent A}{size=\small}

\DeclarePairedDelimiter\abs{\lvert}{\rvert}%
\DeclarePairedDelimiter\norm{\lVert}{\rVert}%
\makeatletter
\let\oldabs\abs
\def\abs{\@ifstar{\oldabs}{\oldabs*}}
\let\oldnorm\norm
\def\norm{\@ifstar{\oldnorm}{\oldnorm*}}
\makeatother
\newcommand*{\Value}{\frac{1}{2}x^2}%

\newcommand{\floatfootnote}[1]{\ifx\[$\else\footnote{#1}\fi}
\newcommand{\floatfootnotes}[1]{\ifx\[$\else\footnote{#1}\fi}



\title{Digital Transformation of Healthcare}
\subtitle{Data Sources}
% \date{\today}
\date{}
\author{Michoel Snow, MD PhD, Glen Ferguson, PhD}
\institute{Center for Health Data Innovations}
% \titlegraphic{\hfill\includegraphics[height=1.5cm]{logo.pdf}}

\begin{document}

\maketitle

\begin{frame}{Objectives}

\end{frame}

\begin{frame}{Ancillary Data Sources}
	\begin{itemize}[<+(0)->]
		\item What other kinds of data can be useful to a study (in addition to those generated by the hospital)?
		\item What are the different ways to get that data?
	\end{itemize}
\end{frame}


\note{
\scriptsize
	\begin{itemize}
		\item Other Data - Weather data, public health records 
		\item Try to find that data live (csv, api, scraping), can you trust the data source?
		\begin{itemize}	
			\scriptsize
			\item Weather - download data, weather underground, wunderweather
			\item Twitter - api
			\item football - scraping (ethics of scraping)
			\item Other - ilinet (flu view) pdfs (new york state influenza surveillance)
		\end{itemize}
	\end{itemize}

}



\begin{frame}{Mobile Health Overview }
	\begin{itemize}[<+(0)->]
		\item What devices can we use?
		\item What information can we get from patients, using these devices?
		\item What information should we get from a moral standpoint?
		\item What information do we want and how often do we want it?
	\end{itemize}
\end{frame}

\note{
\scriptsize
	\begin{itemize}
		\item Phones, computers/internet, alexa/google home, smart watches
		\item passive vs active data collections
	\end{itemize}

}

\begin{frame}{mHealth Usability}
	\begin{itemize}
		\item What elements do you have to consider when designing mHealth applications? 
		\item What are the technical (backend) issues?
	\end{itemize}
\end{frame}

\note{
	\scriptsize
	\begin{itemize}
		\item Some of the main components of usability are - learnability, efficiency/speed, memorability, low error rate, satisfaction	
		\item interoperability, data security, confidentiality, integration with current health care systems
	\end{itemize}
}



\end{document}