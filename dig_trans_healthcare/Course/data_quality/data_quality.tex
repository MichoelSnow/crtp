\documentclass[10pt, xcolor=table]{beamer}

\setbeamertemplate{note page}[default]
%\setbeameroption{hide notes}
\setbeameroption{show notes}
\setbeamerfont{footnote}{size=\tiny}

\usetheme[progressbar=frametitle]{metropolis}
\usepackage{appendixnumberbeamer}

\usepackage{booktabs}
\usepackage[scale=2]{ccicons}

\usepackage{pgfplots}
\usepgfplotslibrary{dateplot}
\usepackage{multicol}
\setlength{\columnsep}{1.5cm}
\usepackage{multirow}


\usepackage{animate}
\usepackage{lmodern}
\usepackage[T1]{fontenc}
\usepackage{mathtools}
\usepackage{graphicx}
\usepackage[font=scriptsize]{caption}
\usepackage{tikz}
\usepackage{stackengine}
\usepackage{array}
\usetikzlibrary{positioning}
\usepackage{tabularx}
\usepackage{tabulary}

\usepackage[math]{cellspace}
\cellspacetoplimit 2pt
\cellspacebottomlimit 2pt


%\definecolor{set1}{RGB}{228, 26, 28}
%\definecolor{set2}{RGB}{77, 175, 74}
%\definecolor{set3}{RGB}{255, 127, 0}
%\definecolor{set4}{RGB}{166, 86, 40}
%\definecolor{set5}{RGB}{153, 153, 153}

\usepackage{xspace}
\newcommand{\themename}{\textbf{\textsc{metropolis}}\xspace}

\newcommand\Fontvi{\fontsize{8}{9}\selectfont}
\newcommand\Fontvr{\fontsize{6}{7}\selectfont}

\setbeamerfont{parent A}{size=\small}

\DeclarePairedDelimiter\abs{\lvert}{\rvert}%
\DeclarePairedDelimiter\norm{\lVert}{\rVert}%
\makeatletter
\let\oldabs\abs
\def\abs{\@ifstar{\oldabs}{\oldabs*}}
\let\oldnorm\norm
\def\norm{\@ifstar{\oldnorm}{\oldnorm*}}
\makeatother
\newcommand*{\Value}{\frac{1}{2}x^2}%

\newcommand{\floatfootnote}[1]{\ifx\[$\else\footnote{#1}\fi}
\newcommand{\floatfootnotes}[1]{\ifx\[$\else\footnote{#1}\fi}



\title{Digital Transformation of Healthcare}
\subtitle{Assessing Data Quality}
% \date{\today}
\date{}
\author{Michoel Snow, MD PhD, Glen Ferguson, PhD}
\institute{Center for Health Data Innovations}
% \titlegraphic{\hfill\includegraphics[height=1.5cm]{logo.pdf}}

\begin{document}

\maketitle

\section{Data Quality}

%\begin{frame}{Factors Which Affect Data Quality}
%	\begin{center}
%		Analysis is only ever as good as the data its built upon.
%	\end{center}
%	\begin{itemize}
%		\item<2-> Data Definition
%		\item<2-> Data Collection
%		\item<2-> Data Processing
%		\item<2-> Data Representation
%	\end{itemize}
%\end{frame}

\begin{frame}{Overview}
	\begin{center}
		Analysis is only ever as good as the data it's built upon.
	\end{center}
	\begin{itemize}[<+(1)->]
		\item What is data quality? What makes data high quality vs low quality?
		\item Where along the process can you affect data quality?
		\item How can you design a study to collect high quality data (Quality assurance)?
		\item How can you identify and correct errors during and after data collection (Quality control)?
	\end{itemize}
\end{frame}

\note{
\scriptsize
	\begin{itemize}
		\item Data quality consists of the objective [accuracy, validity (not outside range of possibilities, all data is for the same pt, formatting requirements, DICOM dates), reliability (dx matches problem list matches coding), legibility (units, shorthand)] and subjective [completeness]
		\item Steps	
		\begin{itemize}
			\scriptsize
			\item Definition/Design - lack of clear definitions for data items/collection, incompatible units, precision, scope, depth
			\item Collection - not enough documentation (drug given/dosage altered but no start and end date), non-adherence to data definitions (collecting data outside of protocol time), human variance/error (bp cuff, RR, incorrect units), Orders are placed (procedures, medications) which are not connected to a rationale or sufficient reason 
			\item Processing - interpretation ('initial' lab, diagnosis date), coding error (mis-entering information such as order of birthdate, or height as 9 cm instead of 90 cm), random (mistyping, illegible handwriting), software errors, Assigning codes to problems treated vs problems tested and ruled out, which complaints do you code/document (doctors as coders)  
		\end{itemize} 
	\end{itemize}

}

\note{
	\scriptsize
	\begin{itemize}
		\item quality assurance - training of personnel (mock exams and reporting), site visits, reduce open-ended questions
		\item quality control - data monitoring (compared to independent source), hand verification, entering data in twice (by different sources), consistency checks
	\end{itemize}
}

%\begin{frame}{Data Collection and Ambiguity}
%	\begin{itemize}
%		\item Diagnosis Date
%		\item Medication Instructions
%		\begin{itemize}
%			\scriptsize
%			\item heparin flush (porcine) in NS Kit 100 unit/mL 500 UNITS. IV QDAY \#30 MILLILITERS
%			\item Heparin Lock Soln 100 unit/mL 500 UNITS. IV BID \#300 MILLILITERS refill x3 flush after each infusion
%			\item Heparin Sodium (Porcine) Syringe 10,000 unit/mL 18000 UNITs SC Q8HR \#180 MLs refill x5
%			\item heparin lock flush (porcine) syrg 100 unit/mL 100 UNITS. IV 3/wk 100 HS PRN \#200 MILLILITERS refi...
%			\item heparin (porcine) in 0.9\% NaCl IV Soln 100 unit/100 mL (1 unit/mL) 1-5 UNITS. IV QDAY \#100 MILLIL...
%		\end{itemize}
%	\end{itemize}
%\end{frame}


\begin{frame}{Extract, Transform and Load (ETL)}
	\begin{itemize}
		\item Extract
		\begin{itemize}
			\item Extract data from source(s)
		\end{itemize}
		\item Transform
		\begin{itemize}
			\item Modify the data for the purposes of analysis or further querying
		\end{itemize}
		\item Load
		\begin{itemize}
			\item Load the data into the target database
		\end{itemize}
	\end{itemize}
\end{frame}


\note{
\scriptsize

}

\begin{frame}{EDW and Clarity}
Walkthrough EDW
\end{frame}

\begin{frame}{Sources}
	\begin{itemize}
		\item \href{http://www.globalhealthworkforce.org/resources/who_improving_data_quality.pdf}{WHO data quality}
		\item \href{https://pdfs.semanticscholar.org/f68b/1f61568a15791dbc6f308ba73d39836fb47e.pdf}{Healthcare Data Warehousing and Quality Assurance}
		\item (2002). Defining and improving data quality in medical registries JAMIA, 9(6), 600-611.
	\end{itemize}

\end{frame}

%\begin{frame}{How Can Data Be Wrong}
%	\begin{itemize}
%		\item Incomplete
%		\item Inconsistent
%		\item Inaccurate
%	\end{itemize}
%\end{frame}

%\begin{frame}{Processes to Assure Data Quality}
%	\begin{itemize}
%		\item Data Provenance
%		\item Sanity Checks
%		\item Exploratory Data Analysis		
%	\end{itemize}
%\end{frame}

\end{document}